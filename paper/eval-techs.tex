\documentclass[a4paper,10pt]{article}
\usepackage[utf8x]{inputenc}
\usepackage{natbib}

%opening
\title{The performance of Iterative Proportional Fitting for spatial microsimulation: applying new tests to an old technique}
\author{Robin Lovelace}

\begin{document}

\maketitle

\section{IPF: theory and applications}
\section{Evaluation techniques}

To verify the integrity of any model, it is necessary to compare its outputs 
with empirical observations or prior expectations gleaned from theory. 
The same principles apply to spatial microsimulation, which can be evaluated using
both internal and \emph{external} validation methods \citep{Edwards2009}. 
\emp{Internal validation} is the process whereby
the aggregate-level outputs of spatial microsimulation are compared with
the input constraint variables. Internal validation typically tackle such issues 
as ``do the simulated populations in each area 
microdata match the total populations implied by the constrain variables?''
and ``what is the level of correspondance between the cell values of different 
attributes in the aggregate input data and the aggregated results of spatial microsimulation?''
As we shall see below, ideally we would hope for a perfect fit between the input
and output datasets during internal evalutation
 
\emph{External validation} is the process whereby the variables that are 
being estimated are compared to data from another source, 
external to the estimation process, so the output dataset is compared with 
another known dataset for those variables.

\section{Input data}
\section{Method: model experiments}
\section{Results}
\section{Conclusions}
\bibliographystyle{model2-names.bst}
\bibliography{/nfs/foe-fs-01_users/georl/Documents/Microsimulation.bib}

\end{document}
