\documentclass[a4paper,10pt]{article}
\usepackage[utf8x]{inputenc}
\usepackage{natbib}
\usepackage{cleveref}
%opening
\title{The performance of Iterative Proportional Fitting for spatial microsimulation: applying new tests to an old technique}
\author{Robin Lovelace}

\begin{document}

\maketitle

\begin{abstract}
Iterative Proportional Fitting (IPF)
is an established technique in Regional Science and
 has been used for a variety of purposes in the field, primary amongst these being 
the generation of small area microdata.  
The technique is mature, widely used and relatively straight-forward
yet there have been few studies focussed on evaluation of it performance. 
An additional research problem is the tendency of 
researchers to `start from scratch', resulting in a variety ad-hock implementations,
with little evidence of the merits of differing approaches.
Although IPF is well described mathematically, reproducible
examples of the algorithm, written in modern programming language, are rare in the academic literature.
These knowledge gaps mean that answers to methodological questions must be guessed:
Under what conditions can a 'perfect' 
match be expected, and how can one test whether these conditions exist in the input datasets?  
How many iterations should be used for any given application? 
What is the impact of initial weights on the final results? 
And what impact will integerisation have on accuracy?  
This paper tackles such questions, using a systematic methodology 
based on publicly available code and input data. 
The results confirm IPF's status as robust and widely applicable 
method for combining areal and individual-level data and demonstrate the importance
of various parameters determining its performace.
We conclude with an agenda for future tests and offer more general guidance on how the 
the method can be optimised to maximise its utility for future research.

Keywords: Iterative proportional fitting, spatial microsimulation, modelling

JEL Code: C
\end{abstract}

\section{Introduction}
The aggregation of survey data has a long history in the social sciences. 
The most common form of 
aggregation conducted by national statistical agencies collecting data about citizens 
is 'flattening': converting data from `long' to `flat' form. This involves involves a) converting all 
variables into discrete categories (e.g.~an individual with an income of £13,000 per year
would be allocated to band of £10,000 to £15,000 per year), b) assigning a column to each category and c) assigning 
integer counts to cells in each column (Fig. 1). Each row in this aggregated form can 
represent the entire dataset or subsets thereof. (Geographical aggregation – one row 
per zone – is the most common form of aggregation, but any subsets, from social class, 
time period or other \emph{cross-tablulations} of the dataset can be used).

The main advantage of 'flat' data is that it can summarise information about a very large number of individuals
in a small number of rows. Often, such flattened datasets are the only format in which 
statistical agencies make large datasets (e.g. the National Census) available, 
to save storage space, ease data analysis and ensure the anonymity and unidentifiability of survey respondents.
The main disadvantage is that the process of flattening is almost always associated with data loss: we do 
not know from a flat dataset if the hypothetical individual mentioned above earns \pounds10,000, \pounds15,000 or 
anywhere in between. Also, when flattening is associated with geographical aggregation, there is usually
a loss of geographical resolution: instead of knowing the person's full postcode from the full dataset, we 
may only be able to pin them down to a relatively large administrative zone such as a Local Authority in the UK context
in the flattened dataset. Finally, the cross-tabulations between the variables are generally during the process of 
flattening.
% make bullet points???

Both types of data representation are common in the physical sciences, although the raw data 
is usually available in long-form. In the social sciences, by contrast, the long format is often unavailable. 
As indicated above, this can be attributed to the long history of and confidentiality 
issues surrounding officially collected survey data. National Census datasets --- in which characteristics of 
every individual are recorded --- and some of the best datasets in social science are only available 
in flat form, however, posing major challenges to researchers in the sciences. 

A related long-standing research problem especially prevalent in geography and regional science
is the scale of analysis. There is often a trade-off between the quality and resolution 
of the data, which can result in a compromise between detail and accuracy of results 
\citep{ballas2003microsimulation-30-years}. A common scenario facing researchers investigating 
a spatial problem is being presented with two datasets: one long aspatial 
survey table where each row represents the characteristics of an individual --- such as
age, sex, employment status and income --- and another of `flat', geographically aggregated 
dataset containing count data for each category. 

As presented in Fig. 1, flat datasets are
are wider and shorter, containing multiple columns for each variable from the raw dataset.
The loss of information about the linkages between
the different variables is common, but not essential, in flat datasets. 
The solution to this problem, however, makes the flat datasets even more unwieldy, as
each new cross tabulation would add $nv1 \times nv2$ additional columns, where 
$nv1$ and $nv2$ are the number of discrete categories in contained in the variables to be
cross tabluted.\footnote{To 
provide a concrete example, a cross-tabulation 
of mode of travel to work by sex and social class would contain 540 (12 * 3 * 15) columns,
due to the number of categories present in each variable. 
Given that there are dozen of additional variables in any large census, this clearly become unwieldy.
} % figure???
The lack of cross tabulation is problematic because 
in many cases information about the linkages between different variables is needed to 
address policy-relevant questions. To pick one example related to 
policies targeted at skilled unemployed young adults: how many unemployed people 
reside in each geographical area who at the same time have a university degree, 
are female and over 25 years old? This question is impossible to answer with a 
geographically aggregated (flat) dataset.

Spatial microsimulation tackles this problem by simulating
the individuals within each administrative zone. The most detailed output of this process of 
spatial microsimulation is `spatial microdata', a long table of individuals with 
associate attributes and weights for each zone (table 1). 
Because these individuals correspond to observations in the survey dataset,
the spatial microdata can be stored with only 3 columns: person identification number (ID), 
zone and weight. Weights are critical to spatial microdata generated in this way, as they 
determine how representative each individual is of each zone. Thus, an even more compact way 
of storing spatial microdata taken from a single dataset is as a `weight matrix' (table 2). 
% the same dataset can be stored using only two variables:  
% small area microdata 
% by combining the small area data with national survey data. The result of this process 
% of population synthesis (in its long % have we already defined long?
%  form --- it can also be represented in a flat table of counts)
%  is a table in which rows represent each 
% individual in the study region, with an additional column indicating the area of residence (Table 1).

Representing the data in this form offers a number of advantages (Clarke and Holm, 1987). These include:
\begin{itemize}
 \item The ability to investigate intra-zone variation (by analysing a the variable after subsetting 
a single area from the data frame),and inter-zone variation in the same dataset.
\item Cross-tabulations can be created for any combination of variables (e.g. age and employment status).
\item The presentation of continuous variables such as income as real numbers, 
rather than as categorical count data, in which there is always data loss.
\item The data is in a form that is ready to be passed into an individual-level model.
For example individual behaviour could simulate traffic, migration or the geographical distribution of future demand for healthcare.
\end{itemize}

% A potential downside of spatial microdata is that it can provide a misleading impression about the 
% quality and detail of the input data. It is important to remember that spatial microdata generated 
% in this way through spatial microsimulation are only as good as the constraint variables and the 
% extent to which the national dataset is representative of the zones in the study area.

These benefits have not gone unnoticed by the spatial modelling community.
Spatial microsimulation is now more than a methodology, and
can be considered as a research field in its
own right, with a growing body of literature, methods and results \citep{Tanton2013}.
In tandem with these methodological developments, a growing number of applied studies
has emerged to take advantage of the new possibilities opened up by such small area individual-level data,
with applications as diverse as the location of health services \citep{Tomintz2008} to the
analysis of commuting \citep{Lovelace2014-jtg}.
Various strategies for generating small area microdata are available,
as described in recent overviews on the subject \citep{Tanton2013, Ballas2013-4policy-analysis, Hermes2012a}.
However, it is not always clear which one is most appropriate,
as highlighted in the conclusions of a recent Reference Guide to spatial microsimulation
\citep[p~270]{Clarke2013-concs}:
``It would be desirable if the spatial microsimulation community were able to continue to
analyse which of the various reweighting/synthetic reconstruction techniques is most accurate
--- or to identify whether one approach is superior for some applications while another
approach is to be preferred for other applications''.

The longest-established and most straight-forward of these is Iterative Proportional Fitting (IPF), the performance of which is the subject of this paper...





\section{IPF: theory and applications}
Iterative Proportional Fitting has a long history in Statistics and the social
sciences (especially in Economics and Geography).
In statistical language, IPF is used to estimate the values of internal
cells in a cross-tabulated table to fit known margins.
The method is also known as matrix raking or scaling,
biproportional fitting, RAS, and entropy maximising.
There are a number of texts discussing the method in a mathematical formal
way in some detail (for example \citealp{Mosteller1968, Fienberg1970, Pritchard2012}).
\citep{Birkin1988} provide a succinct presentation of the
mathematics of IPF and their work is the foundation of
the work presented here...% described in origin; rethink this...???

\section{Evaluation techniques}

To verify the integrity of any model, it is necessary to compare its outputs
with empirical observations or prior expectations gleaned from theory.
The same principles apply to spatial microsimulation, which can be evaluated using
both internal and \emph{external} validation methods \citep{Edwards2009}.
\emph{Internal validation} is the process whereby
the aggregate-level outputs of spatial microsimulation are compared with
the input constraint variables. Internal validation typically tackle such issues
as ``do the simulated populations in each area
microdata match the total populations implied by the constrain variables?''
and ``what is the level of correspondence between the cell values of different
attributes in the aggregate input data and the aggregated results of spatial microsimulation?''
As we shall see below, ideally we would hope for a perfect fit between the input
and output datasets during internal evaluation.
\emph{External validation} is the process whereby the variables that are
being estimated are compared to data from another source,
external to the estimation process, so the output dataset is compared with
another known dataset for those variables.
%... ?

This section outlines the evaluation techniques that are commonly used in the
field, with a strong focus on internal validation. The options for external
validation are heavily context dependent so must generally be decided on a
case-by-case basis, and are mentioned in passing with a focus on general principles
rather than specific advice. The methods are presented in ascending order of complexity
and roughly descending order of frequency of use, ranging from Pearson's coefficient of
correlation to entropy based measures. Before these methods are described, it is
worth stating the the purpose of this section is not to find the `best' evaluation metric:
each has advantages and disadvantages, the importance of which will vary depending on the
nature of the research. A final point is that the use of a variety of techniques in this paper
is of interest in itself. The high degree of correspondence between them (presented in \cref{cresults}),
suggests that researchers need only to present one or two metrics (but should
perhaps try more, for corroboration) to establish relative levels of goodness of fit between
different model configurations.

\subsection{Scatter plots and Pearson's coefficient of correlation}
A scatter plot of cell counts for each category for the original and simulated variables is
a basic but very useful preliminary diagnostic tool in spatial microsimulation
\citep{Ballas2005;Edwards2009}.
In addition marking the points, depending on the
variable and category % have these been defined?
which they represent, can help identify which variables are particularly problematic,
aiding the process of improving faulty code (sometimes referred to as `debugging').

In these scatter plots each data point represents a single zone-category
combination (e.g. 16-25 year old female in zone 001), with the x axis value corresponding
to the number of individuals of that description in the input constraint table.
The y value corresponds to the
aggregated count of individuals in the same category in the aggregated
(flat) representation of the spatial microdata output from the model. % have we defined spatial microdata?
This stage can be conducted either before or after \emph{integerisation}
(more on this below). As taught in basic statistics courses, the closer the points to the
1:1 (45 degree, assuming the axes have the same scale) line, the better the fit.

Pearson's coefficient of correlation (henceforth Pearson's $r$) is a quantitative
indicator of how much deviation there is from this ideal 1:1 line. It varies from 0 to 1,
and in this context reveals how closely the simulated values fit the actual (census) data.
An $r$ value of 1 represents a perfect fit; an $r$ value close to zero suggests no correspondence
between then constraints and simulated outputs (i.e. that the model has failed).
We would expect to see very high coefficient of determination for
internal validation (the constraint) and strong
positive correlation between target variables that have been estimated and
external datasets (e.g. income).
% However, regression analysis does not give any information about the fit of the simulated data to the ‘ideal’ line (i.e. where y  =  x and the simulated data is the same as the actual data). Rather, R 2 expresses the fit of the data to the ‘best fit’ line through that data. That is, the coefficient of determination is providing information about precision, not accuracy (Edwards and Tanton, 2013).

% One way to do this is with a t test. With a spatial microsimulation model validation, the data are paired (given we are comparing simulated with actual data), thus an equal variance 2-tailed t test can be used to determine if there is any significant difference between the two datasets (i.e. simulated and actual). Thus, if the simulation is robust, we would expect to see no significant differences between the simulated and actual values for the input variables (and estimated/output variables, if known data are available). This enables the model accuracy to be assessed, as opposed to simply its precision.

\subsection{Standard error around identity (SEI)}
here SEI is the standard error around identity, y are the estimated values for
each area, y rel are the reliable estimates for each area from a census or other
data source and y rel is the mean estimate for all areas where reliable data are available.
This estimate has been used in validation by both Ballas and Tanton (see Ballas et al. 2007; Tanton et al. 2011).

\subsection{Total Absolute and Standardized Error (TAE and SAE)}
The SAE addresses this issue by using the population size as the denominator,
 but generally, authors seem to use total population, rather than the population
for that categorization of the variable, which may be deemed to understate the size
of the errors. Also, this measure does not provide any information on whether any
 differences are statistically significant (Edwards and Tanton, 2013).

\subsection{Z scores}
Here, the evaluation mainly takes place at the cell level (for a single constraint in a single area).
However, there should also be attention for the overall picture like spatial concentration.

% Spatial indicators etc??? Add equations to above methods soooon

\section{Input data}
Three input datasets form the basis of the scenarios tested in this paper,
`simple', `small-area' and `sheffield'. Each consists of a set of aggregate
constraints and corresponding individual-level survey tables.
The latter are anonymous and have been `scrambled' to enable their
publication on the internet. This enables the results to be replicated by others,
without breaching the user licenses of the data.
To this end, all the input data (and code) used to generate the results reported
in the paper are available online on the code-sharing site GitHub.
The simplest scenario consists of 6 zones which must be populated by a
sample survey dataset of only five individuals, constrained by two variables.
This scenario is purely hypothetical and is used to represent the IPF procedure
and the tests in the simplest possible terms. Its small size allows the entirety
of the input data to be viewed in a couple of tables that fit on a single page
(see \cref{t1} and \cref{t2}). As with all data used in this paper, these tables can be
viewed and downloaded online.

\begin{table}[htbp]
\caption{Individual-level data for the simple scenario. Available on github from here.}
\begin{tabular}{rrl}

\multicolumn{1}{l}{Id} & \multicolumn{1}{l}{Age} & Sex \\
1 & 59 & m \\
2 & 54 & m \\
3 & 35 & m \\
4 & 73 & f \\
5 & 49 & f \\
\end{tabular}
\label{t1}
\end{table}

\begin{table}[htbp]
\caption{Constraint variables for the simple scenario, also available online.}
\begin{tabular}{rrrrr}

\multicolumn{1}{l}{} & \multicolumn{1}{c}{Age constraint} & \multicolumn{1}{l}{} & \multicolumn{1}{c}{Sex constraint} & \multicolumn{1}{l}{} \\
\multicolumn{1}{l}{} & \multicolumn{ 2}{l}{16 – 49} & \multicolumn{ 2}{l}{m} \\
1 & 8 & 4 & 6 & 6 \\
2 & 2 & 8 & 4 & 6 \\
3 & 7 & 4 & 3 & 8 \\
4 & 5 & 4 & 7 & 2 \\
5 & 7 & 3 & 6 & 4 \\
6 & 5 & 3 & 2 & 6 \\
\end{tabular}
\label{t2}
\end{table}



\section{Method: model experiments}
\section{Results}
\label{cresults}

\section{Conclusions}
\bibliographystyle{model2-names.bst}
\bibliography{/home/robin/Documents/powerstarrefs/Microsimulation.bib}

\end{document}
