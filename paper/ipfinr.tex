\documentclass[a4paper,10pt]{article}
\usepackage[utf8x]{inputenc}
\usepackage{natbib}

%opening
\title{The performance of Iterative Proportional Fitting for spatial microsimulation: applying new tests to an old technique}
\author{Robin Lovelace}

\begin{document}

\maketitle

\begin{abstract}
Iterative Proportional Fitting (IPF)
is an established technique in Regional Science and
 has been used for a variety of purposes in the field, primary amongst these being 
the generation of small area microdata.  
The technique is mature, widely used and relatively straight-forward
yet there have been few studies focussed on evaluation of it performance. 
An additional research problem is the tendency of 
researchers to `start from scratch', resulting in a variety ad-hock implementations,
with little evidence of the merits of differing approaches.
Although IPF is well described mathematically, reproducible
examples of the algorithm, written in modern programming language, are rare in the academic literature.
These knowledge gaps mean that answers to methodological questions must be guessed:
Under what conditions can a 'perfect' 
match be expected, and how can one test whether these conditions exist in the input datasets?  
How many iterations should be used for any given application? 
What is the impact of initial weights on the final results? 
And what impact will integerisation have on accuracy?  
This paper tackles such questions, using a systematic methodology 
based on publicly available code and input data. 
The results confirm IPF's status as robust and widely applicable 
method for combining areal and individual-level data and demonstrate the importance
of various parameters determining its performace.
We conclude with an agenda for future tests and offer more general guidance on how the 
the method can be optimised to maximise its utility for future research.

Keywords: Iterative proportional fitting, spatial microsimulation, modelling

JEL Code: C
\end{abstract}

\section{Introduction}
The aggregation of survey data has a long history in the social sciences. 
The most common form of 
aggregation conducted by national statistical agencies collecting data about citizens 
is 'flattening': converting data from `long' to `flat' form. This involves involves a) converting all 
variables into discrete categories (e.g.~an individual with an income of £13,000 per year
would be allocated to band of £10,000 to £15,000 per year), b) assigning a column to each category and c) assigning 
integer counts to cells in each column (Fig. 1). Each row in this aggregated form can 
represent the entire dataset or subsets thereof. (Geographical aggregation – one row 
per zone – is the most common form of aggregation, but any subsets, from social class, 
time period or other \emph{cross-tablulations} of the dataset can be used).

The main advantage of 'flat' data is that it can summarise information about a very large number of individuals
in a small number of rows. Often, such flattened datasets are the only format in which 
statistical agencies make large datasets (e.g. the National Census) available, 
to save storage space, ease data analysis and ensure the anonymity and unidentifiability of survey respondents.
The main disadvantage is that the process of flattening is almost always associated with data loss: we do 
not know from a flat dataset if the hypothetical individual mentioned above earns \pounds10,000, \pounds15,000 or 
anywhere in between. Also, when flattening is associated with geographical aggregation, there is usually
a loss of geographical resolution: instead of knowing the person's full postcode from the full dataset, we 
may only be able to pin them down to a relatively large administrative zone such as a Local Authority in the UK context
in the flattened dataset. Finally, the cross-tabulations between the variables are generally during the process of 
flattening.
% make bullet points???

Both types of data representation are common in the physical sciences, although the raw data 
is usually available in long-form. In the social sciences, by contrast, the long format is often unavailable. 
As indicated above, this can be attributed to the long history of and confidentiality 
issues surrounding officially collected survey data. National Census datasets --- in which characteristics of 
every individual are recorded --- and some of the best datasets in social science are only available 
in flat form, however, posing major challenges to researchers in the sciences. 

A related long-standing research problem especially prevalent in geography and regional science
is the scale of analysis. There is often a trade-off between the quality and resolution 
of the data, which can result in a compromise between detail and accuracy of results 
\citep{ballas2003microsimulation-30-years}. A common scenario facing researchers investigating 
a spatial problem is being presented with two datasets: one long aspatial 
survey table where each row represents the characteristics of an individual --- such as
age, sex, employment status and income --- and another of `flat', geographically aggregated 
dataset containing count data for each category. 

As presented in Fig. 1, flat datasets are
are wider and shorter, containing multiple columns for each variable from the raw dataset.
The loss of information about the linkages between
the different variables is common, but not essential, in flat datasets. 
The solution to this problem, however, makes the flat datasets even more unwieldy, as
each new cross tabulation would add $nv1 \times nv2$ additional columns, where 
$nv1$ and $nv2$ are the number of discrete categories in contained in the variables to be
cross tabluted.\footnote{To 
provide a concrete example, a cross-tabulation 
of mode of travel to work by sex and social class would contain 540 (12 * 3 * 15) columns,
due to the number of categories present in each variable. 
Given that there are dozen of additional variables in any large census, this clearly become unwieldy.
} % figure???
The lack of cross tabulation is problematic because 
in many cases information about the linkages between different variables is needed to 
address policy-relevant questions. To pick one example related to 
policies targeted at skilled unemployed young adults: how many unemployed people 
reside in each geographical area who at the same time have a university degree, 
are female and over 25 years old? This question is impossible to answer with a 
geographically aggregated (flat) dataset.

Spatial microsimulation tackles this problem by simulating
the individuals within each administrative zone. The most detailed output of this process of 
spatial microsimulation is `spatial microdata', a long table of individuals with 
associate attributes and weights for each zone (table 1). 
Because these individuals correspond to observations in the survey dataset,
the spatial microdata can be stored with only 3 columns: person identification number (ID), 
zone and weight. Weights are critical to spatial microdata generated in this way, as they 
determine how representative each individual is of each zone. Thus, an even more compact way 
of storing spatial microdata taken from a single dataset is as a `weight matrix' (table 2). 
% the same dataset can be stored using only two variables:  
% small area microdata 
% by combining the small area data with national survey data. The result of this process 
% of population synthesis (in its long % have we already defined long?
%  form --- it can also be represented in a flat table of counts)
%  is a table in which rows represent each 
% individual in the study region, with an additional column indicating the area of residence (Table 1).

Representing the data in this form offers a number of advantages (Clarke and Holm, 1987). These include:
\begin{itemize}
 \item The ability to investigate intra-zone variation (by analysing a the variable after subsetting 
a single area from the data frame),and inter-zone variation in the same dataset.
\item Cross-tabulations can be created for any combination of variables (e.g. age and employment status).
\item The presentation of continuous variables such as income as real numbers, 
rather than as categorical count data, in which there is always data loss.
\item The data is in a form that is ready to be passed into an individual-level model.
For example individual behaviour could simulate traffic, migration or the geographical distribution of future demand for healthcare.
\end{itemize}

A potential downside of spatial microdata is that it can provide a misleading impression about the 
quality and detail of the input data. It is important to remember that spatial microdata generated 
in this way through spatial microsimulation are only as good as the constraint variables and the 
extent to which the national dataset is representative of the zones in the study area.







\section{IPF: theory and applications}
\section{Evaluation techniques}
\section{Input data}
\section{Method: model experiments}
\section{Results}
\section{Conclusions}
\bibliographystyle{model2-names.bst}
\bibliography{/nfs/foe-fs-01_users/georl/Documents/Microsimulation.bib}

\end{document}
